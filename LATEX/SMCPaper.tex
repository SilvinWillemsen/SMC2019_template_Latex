% -----------------------------------------------
% Template for SMC 2019
% adaed from the template for SMC 2018
% -----------------------------------------------

\documentclass{article}
\usepackage{smc2019}
\usepackage{times}
\usepackage{ifpdf}
\usepackage[english]{babel}
\usepackage{cite}
\usepackage{physics}

%%%%%%%%%%%%%%%%%%%%%%%% Some useful packages %%%%%%%%%%%%%%%%%%%%%%%%%%%%%%%
%%%%%%%%%%%%%%%%%%%%%%%% See related documentation %%%%%%%%%%%%%%%%%%%%%%%%%%
\usepackage{amsmath} % popular packages from Am. Math. Soc. Please use the 
\usepackage{amssymb} % related math environments (split, subequation, cases,
\usepackage{amsfonts}% multline, etc.)
\usepackage{bm}      % Bold Math package, defines the command \bf{}
%\usepackage{paralist}% extended list environments
%%subfig.sty is the modern replacement for subfigure.sty. However, subfig.sty 
%%requires and automatically loads caption.sty which overrides class handling 
%%of captions. To prevent this problem, preload caption.sty with caption=false 
%\usepackage[caption=false]{caption}
%\usepackage[font=footnotesize]{subfig}


%user defined variables
\def\papertitle{Real-time control of advanced Physical Models using the Sensel Morph}
\def\firstauthor{Silvin Willemsen}
\def\secondauthor{Nikolaj Andersson}
\def\thirdauthor{Stefania Serafin}
\def\fourthauthor{Stefan Bilbao}

% adds the automatic
% Saves a lot of output space in PDF... after conversion with the distiller
% Delete if you cannot get PS fonts working on your system.

% pdf-tex settings: detect automatically if run by latex or pdflatex
\newif\ifpdf
\ifx\pdfoutput\relax
\else
   \ifcase\pdfoutput
      \pdffalse
   \else
      \pdftrue
\fi

\ifpdf % compiling with pdflatex
  \usepackage[pdftex,
    pdftitle={\papertitle},
    pdfauthor={\firstauthor, \secondauthor, \thirdauthor, \fourthauthor},
    bookmarksnumbered, % use section numbers with bookmarks
    pdfstartview=XYZ % start with zoom=100% instead of full screen; 
                     % especially useful if working with a big screen :-)
   ]{hyperref}
  %\pdfcompresslevel=9

  \usepackage[pdftex]{graphicx}
  % declare the path(s) where your graphic files are and their extensions so 
  %you won't have to specify these with every instance of \includegraphics
  \graphicspath{{./figures/}}
  \DeclareGraphicsExtensions{.pdf,.jpeg,.png,.eps}

  \usepackage[figure,table]{hypcap}

\else % compiling with latex
  \usepackage[dvips,
    bookmarksnumbered, % use section numbers with bookmarks
    pdfstartview=XYZ % start with zoom=100% instead of full screen
  ]{hyperref}  % hyperrefs are active in the pdf file after conversion

  \usepackage[dvips]{epsfig,graphicx}
  % declare the path(s) where your graphic files are and their extensions so 
  %you won't have to specify these with every instance of \includegraphics
  \graphicspath{{./figures/}}
  \DeclareGraphicsExtensions{.eps}

  \usepackage[figure,table]{hypcap}
\fi

%setup the hyperref package - make the links black without a surrounding frame
\hypersetup{
    colorlinks,%
    citecolor=black,%
    filecolor=black,%
    linkcolor=black,%
    urlcolor=black
}


% Title.
% ------
\title{\papertitle}

% Authors
% Please note that submissions are NOT anonymous, therefore 
% authors' names have to be VISIBLE in your manuscript. 
%
% Single address
% To use with only one author or several with the same address
% ---------------
% \twoauthors
%   {Silvin Willemsen, Nikolaj Andersson and Stefania Serafin} { \\ Multisensory Experience Lab, CREATE, Aalborg University Copenhagen \\ %
%     {\tt \href{mailto:sil@create.aau.dk}{\{sil, nsa, sts\}@create.aau.dk}}}

% Two addresses
% --------------
\twoauthors
  {\firstauthor, \secondauthor, \thirdauthor} {Multisensory Experience Lab,\\
   Aalborg University Copenhagen \\
   Copenhagen, Denmark\\ %
    {\tt \href{mailto:sil@create.aau.dk}{\{sil, nsa, sts\}@create.aau.dk}}}
  {\fourthauthor} {Acoustics and Fluid Dynamics Group/Music, \\
  University of Edinburgh \\
  Edinburgh, UK \\ %
    {\tt \href{mailto:sbilbao@staffmail.ed.ac.uk}{sbilbao@staffmail.ed.ac.uk}}}

% Three addresses
% --------------
%   {\firstauthor} {{} %
%       {}}
%   {\secondauthor} {{Multisensory Experience Lab, CREATE, Aalborg University Copenhagen}
%      {\tt \href{mailto:sil@create.aau.dk}{\{sil, nsa, sts\}@create.aau.dk}}}
%   {\thirdauthor}{
%      {}}
    % {\fourthauthor} { Affiliation4 \\ %
    %  {\tt \href{mailto:author3@smcnetwork.org}{author3@smcnetwork.org}}}


% ***************************************** the document starts here ***************
\begin{document}
%
\capstartfalse
\maketitle
\capstarttrue
%
\begin{abstract}
Lorum Ipsum
\end{abstract}
%

\section{Introduction}\label{sec:introduction}
The behaviour of musical instruments can be well defined by partial differential equations (PDEs) \cite{Bilbao2018:Tutorial}.

Finite-difference schemes (FDSs)

The physical models (PMs) used as a case study in this project are the stiff string and the plate.

On top of all this, we have used the expressive Sensel Morph \cite{sensel2018} control surface to control the PMs in real-time, something that to the best of the authors' knowledge has not been done before. 

This paper is structured as follows: Section \ref{sec:PDE} describes the PMs used in the implementation and Section \ref{sec:FDS} shows the FDSs used to digitally implement these models. Furthermore, Section \ref{sec:energy} will describe an energy analysis of a connected string-plate system, Section \ref{sec:implementation} will show how to implement the FDSs, Section \ref{sec:instruments} shows several different configurations of PMs inspired by real musical instruments and finally Section \ref{sec:discussion} and Section \ref{sec:conclusion} will discuss and conclude upon the work shown in this paper.

\section{Models}\label{sec:PDE}
In this section, the partial differential equations for the damped stiff string and the plate will be presented. 

% \begin{equation}
%     \pdv[2]{u}{t} = \gamma^2 \pdv[2]{u}{x} -\kappa^2 \pdv[4]{u}{x} - 2\sigma_0\pdv{u}{t} + 2\sigma_1\frac{\partial^3u}{\partial tx^2}
% \end{equation}

\subsection{Stiff string}\label{subsec:stiffStringPDE}
The state $u = u(x,t)$ describes the transverse displacement of the string. The subscript for $u$ denotes a single derivative with respect to time $t$ or space $x$ respectively. The partial differential equation for the damped stiff string is defined as \cite{Bilbao2009:NumericalSoundSynthesis} 
\begin{equation}\label{eq:stiffString}
    u_{tt} = \gamma^2 u_{xx}-\kappa^2u_{xxxx} - 2\sigma_0u_{t} + 2\sigma_1u_{txx},
\end{equation}
where $\gamma = c/L$ is wave-speed with units of frequency[s$^{-1}$], $\kappa = \sqrt{EI/\rho AL^4}$ is a stiffness parameter
%[$\sqrt{\text{kg}}\cdot$m$^{-2}\cdot$ s$^{-2}$]
and $\sigma_0 \geq 0$ and $\sigma_1 \geq 0$ are frequency-dependent and frequency-independent damping respectively.

We can add an excitation term to extend Equation \eqref{eq:stiffString} to a bowed string \cite{Bilbao2009:NumericalSoundSynthesis} 
\begin{align}
    \label{eq:bowedString} &u_{tt} = ... - \delta(x-x_\text{B})F_\text{B}\phi(v_\text{rel}), \quad \text{with} \\
    &v_\text{rel} = u_t(x_\text{B}) - v_\text{B}
\end{align}
%the $\delta$ function is only non-zero at the bowing point, effectively locating the bowing interaction.  
where $F_\text{B} = f_\text{B}/ \rho AL$ is the excitation function [m/s$^2$] with bowing force $f_\text{B}$ [N], material density $\rho$ [kg/m$^3$], cross-sectional area $A$ [m$^2$] and string length $L$ [m]. The relative velocity $v_\text{rel}$ is defined as the difference between the velocity of the string at bowing point $x_\text{B}$ and the bowing velocity $v_\text{B}$ [m/s] and $\phi$ is a friction characteristic, which has been chosen to be \cite{Bilbao2009:NumericalSoundSynthesis}
\begin{equation}
    \phi(v_\text{rel}) = \sqrt{2a}v_\text{rel} e^{-av_\text{rel}^2+1/2}.
\end{equation}
Furthermore,
\begin{equation} \label{eq:dirac}
    \delta(x-x_\text{B}) =
\begin{cases}
    1, & \text{if } x=x_\text{B}\\
    0,              & \text{otherwise},
\end{cases}
\end{equation}
is referred to as the spatial Dirac delta function, which, when multiplied onto the excitation, applies it only to the point $x_\text{B}$ on the string.
\subsection{Plate}\label{subsec:platePDE}
In the case of a plate, the state $u = u(x,y,t)$ is now defined over two spatial dimensions. The PDE for a damped plate is \cite{Bilbao2009:NumericalSoundSynthesis}
\begin{equation}\label{eq:platePDE}
    u_{tt} = -\kappa^2 \Delta\Delta u - 2 \sigma_0 u_{t} + 2\sigma_1 \Delta u_{t},
\end{equation}
where $\kappa = \sqrt{Eh^2/12\rho(1-\nu)}/L_xL_y$ is again a stiffness parameter with Young's modulus $E$ [N/m$^2$], plate-thickness $h$ [m], material density $\rho$ [kg/m$^3$], unit-less Poisson's ratio $\nu = 0.3$, and horizontal and vertical length of the plate $L_x$ and $L_y$ [m]. Furthermore, $\Delta$ represents the 2D Laplacian (also see Equation \eqref{eq:2D}). Just like in the case of the string, an extra term can be added as an input:
\begin{equation}\label{eq:plateExcitation}
    u_{tt} = ... + F_\text{e}E_\text{e},
\end{equation}
where $F_\text{e} = f_\text{e} / \rho h L_xL_y $ with excitation force $f_\text{e}$ [N] and $E_\text{e}$ are an excitation function and the excitation area respectively. 

\subsection{Connections}\label{sec:connections}
Adding connections between different PMs, further referred to as elements, adds another term to Equation \eqref{eq:bowedString} or \eqref{eq:plateExcitation}
%% this paragraph is too wide for some reason
\begin{align}
    u_{tt} &= ... + F_\alpha E_\alpha, \\
    u_{tt} &= ... + F_\beta E_\beta,
\end{align}
where $F_\alpha$ and $F_\beta$ are the forces of the connection at connection areas $E_\alpha$ and $E_\beta$ respectively. If the a connection area consists of only one point, $E$ reduces to $\delta(x-x_c)$ where $x_c$ is the point of connection. We use the implementation as presented in \cite{Bilbao2009:ModularPercussion} where the connection between two elements is a non-linear spring. The forces it imposes on the elements it connects - denoted by $\alpha$ and $\beta$ - are defined as
\begin{align}
    F_\alpha &= -\omega_0^2\eta - \omega_1^4\eta^3 - 2\sigma_\times\eta_t,\\
    F_\beta &= -M_{\alpha/\beta}F_\alpha,
\end{align}
where $\omega_0$ and $\omega_1$ are the linear and non-linear spring coefficients respectively, $\sigma_\times$ is the damping factor, $M_{\alpha/\beta}$ is the mass ratio between the two elements and $\eta$ is the relative displacement between the connected elements at the point of connection. The subscript $t$ again denotes a time derivative.


\section{Finite-Difference Schemes}\label{sec:FDS}
To be able to digitally implement the continuous equations shown in the previous section, they need to be approximated. The models can be discretised at times $t = nk$, where $n \in \mathbb{W}$ and $k = 1 / f_\text{s}$ is the time-step with sample-rate $f_\text{s}$ and locations $x = lh$, where $l \in [0,N]$ with $N$ being the total number of points and $h$ is the grid-spacing of the model which is calculated differently for each model (see sub-sections below). The discretised variable $u_l^n$ is $u(x,t)$ at the $n$th time step and the $l$th point on the string. In the case of a plate, the second spatial dimension is discretised using $x = lh$ where $l \in [0,N_x]$ with $N_x$ being the total horizontal number of points and $y = mh$ where $m \in [0,N_y]$ with $N_y$ being the total vertical number of points. 
Approximations for the derivatives in the equations found in Section \ref{sec:PDE} are described in the following way: 

When approximating the PDEs shown in Section \ref{sec:PDE}, we use operators 
% \begin{equation}
    \begin{align}
        \label{eq:secondSpacex}\delta_{xx}u_l^n &= \frac{1}{h^2}\big(u_{l+1}^n - 2u_l^n + u_{l-1}^n\big),\\
        % \label{eq:secondSpacey}\delta_{yy}u_l^n &= \frac{1}{h^2}\big(u_{m+1}^n - 2u_m^n + u_{m-1}^n\big),\\
        %  \label{eq:fourthSpace}\delta_xxxx &= \frac{1}{h^4}\big(u_{l+2}^n - 4u_{l+1}^2 + 6 u_{l}^n - 4u_{l-1}^n + u_{l-2}^n\big),\\
        \label{eq:backwardsTime}\delta_{t-} u^n_l &= \frac{1}{k}\big(u_l^{n}-u_l^{n-1}\big),\\
        \label{eq:centerTime}\delta_{t\cdot} u^n_l &= \frac{1}{2k}\big(u_l^{n+1}-u_l^{n-1}\big),\\
        \label{eq:secondTime}\delta_{tt}u_l^n &= \frac{1}{k^2} \big(u_l^{n+1} - 2u_l^n + u_l^{n-1}\big),\\
        % u_{txx} &= \frac{1}{hk^2}\big(u_{l+1}^n - 2u_l^2 + u_{l-1}^n - u_{l+1}^{n-1} + 2u_l^{n-1} - u_{l-1}^{n-1}\big),\\
        % \Delta u &= \frac{1}{h^2}(u_{l, m+1}^{n} + u_{l, m-1}^{n}
        % + u_{l+1, m}^{n} + u_{l-1, m}^{n} - 4u_{l, m}^{n}),\\
         \label{eq:2D}\delta_\Delta u_{l,m}^n &= \frac{1}{h^2}\big(u_{l,m+1}^n+u_{l,m-1}^n+u_{l+1,m}^n+ \\
         & \qquad \quad u_{l-1,m}^n - 4u_{l,m}^n\big).
            % \Delta \Delta u &= \frac{1}{h^2}(\Delta u_{l, m+1}^{n} + \Delta u_{l, m-1}^{n},\\
            % & \qquad\quad + \Delta u_{l+1, m}^{n} + \Delta u_{l-1, m}^{n} - 4\Delta u_{l, m}^{n}),\\
        %   \label{eq:biharmonic}   \Delta \Delta u &= u_{xxxx} + 2u_{xxyy} + u_{yyyy}.
    \end{align}
% \end{equation}

\subsection{Stiff String}\label{subsec:stiffStringFDS}
Equation \eqref{eq:bowedString} can be approximated using
\begin{equation}
\begin{aligned}\label{eq:stiffStringFDS}
\delta_{tt} u_l^n =&\gamma^2 \delta_{xx} u_l^n -\kappa^2\delta_{xx}\delta_{xx} u_l^n - 2\sigma_0\delta_{t\cdot} u_l^n  \\
&+ 2\sigma_1\delta_{t-}\delta_{xx}u_l^n - \delta_{x_\text{B}}F_\text{B}\phi(v_\text{rel}),
\end{aligned}
\end{equation}
where $\delta_{x_\text{B}} = \delta(x-x_\text{B})$ (see Equation \eqref{eq:dirac}) and 
\begin{equation}\label{eq:descreteRelativeVel}
    v_\text{rel} = \delta_{t\cdot}u(x_\text{B})-v_\text{B}.
\end{equation}
For our implementation, clamped boundary conditions were used, defined as:
\begin{equation}\label{boundary}
    u = u_x = 0 \quad \text{where} \quad l = \{0, N\}.
  \end{equation}
For stability reasons, the grid-spacing needs to abide the following condition
\begin{equation}
    h \geq \sqrt{\frac{\gamma^2 k^2 + 4 \sigma_1 k + \sqrt {(\gamma^2 k^2 + 4 \sigma_1 k)^2 + 16 \kappa^2 k^2}}{2}}.
\end{equation}
The closer $h$ is to this limit, the higher the quality of the implementation. The number of points $N$ can then be calculated using 
\begin{equation}
    N = h^{-1}.
\end{equation}


\subsection{Plate}
  

Equation \eqref{eq:plateExcitation} can be approximated using
\begin{equation}
    \begin{aligned}\label{eq:plateFDS}
        \delta_{tt}u_{l,m}^n = &-\kappa^2 \delta_\Delta\delta_\Delta u_{l,m}^n - 2\sigma_0\delta_{t\cdot}u_{l,m}^n \\
        &+ 2\sigma_1\delta_{t−}\delta_\Delta u_{l,m}^n + F_\text{e}E_\text{e}
    \end{aligned}
\end{equation}
In the case of the plate, we set the number of horizontal and vertical points and calculate grid spacing $h$ from that using 

\begin{equation}
    h = \frac{\sqrt{N_x / N_y}}{N_x}.
\end{equation}

\subsection{Connections}
\begin{align}
    F_\alpha &= -\omega_0^2\mu_{t\cdot}\eta - \omega_1^4\eta^2\mu_{t\cdot}\eta - 2\sigma_\times\delta_{t\cdot}\eta,\\
    F_\beta &= -M_{\alpha/\beta}F_\alpha,
\end{align}

The relative displacement $\eta$ between $\alpha$ and $\beta$ can be calculated as
\begin{equation}
    \eta^n = h_\alpha u_{\alpha, x_\alpha}^n - h_\beta u_{\beta,x_\beta}^n,
\end{equation}
which, in other words, is the difference between the state of element $\alpha$ at connection point $x_\alpha$ and the state of element $\beta$ at connection point $x_\beta$ scaled by their respective grid-spacings $h_\alpha$ and $h_\beta$.


\section{Energy Analysis}\label{sec:energy}

\begin{equation}
\begin{aligned}
    &\mathfrak{H}\\
    &\mathfrak{B}\\
    &\mathfrak{T}\\
    &\mathfrak{D}
\end{aligned}
\end{equation}

% put the figure here so it appears on the next page
\begin{figure*}[t]
    \centering
    \includegraphics[width=2.1\columnwidth]{FullPlate}
    \caption{A visualisation of the finite-difference scheme of the plate (also see Equation \eqref{eq:plateImplementation} in Appendix A). The dots and equations represent the locations $(l,m)$ and what these are multiplied with. (a) An overview. (b) The current time-step $n$. (c) The previous time-step $n-1$. \label{fig:plateFDS}}
 \end{figure*}
\section{Implementation}\label{sec:implementation}
%kind of a methods section
In this section, we will present how to implement the FDSs presented in Section \ref{sec:FDS} and elaborate more on the parameters used. The values for most parameters have been arbitrarily chosen and can - as long as they satisfy the conditions - be changed. We used C++ along with the JUCE framework for implementing the PMs and connections in real-time. The main hardware used for testing was a MacBook Pro with a 2.2 GHz Intel Core i7 processor.

\textit{Note: In this paper we have used the simple case of a single point for bowing, excitation and connections. These can be extended to a bowing area, excitation area and area of connection. For more information on this, we would like to refer the reader to \cite{Bilbao2009:ModularPercussion}}.

\subsection{String}
In order to implement the FDSs presented in Section \ref{sec:FDS} they need to be solved for $u^{n+1}$. Equation \eqref{eq:stringImplementation} found in Appendix A shows a solved finite-difference scheme of Equation \eqref{eq:stiffStringFDS}.
The wave-speed of the string is proportional to the fundamental frequency of the stiff string according to
\begin{equation}
    \gamma = 2 f_0.
\end{equation}
This can be altered in real-time to control the pitch.
The stiffness parameters have been chosen to be $\sigma_0 = 0.1$ and $\sigma_1 = 0.005$. 

% We implemented a damping point in the model that acts as a finger on the (virtual) neck of the instrument controlling pitch. After $u^{n+1}$ is calculated, the following operation is performed at the point of damping:

% \begin{equation}
%     u_{x_\text{f}}^{n+1} = u_{x_\text{f}}^{n+1}  \sigma_\text{f},
% \end{equation}
% where $\sigma_\text{f} \in [0,1]$ is the damping coefficient of the finger. 
The stiffness can be calculated using
\begin{equation}
    \kappa = \frac{\sqrt{B}\gamma}{\pi},
\end{equation}
where $B = 0.001$ is the inharmonicity coefficient [m$^{-2}$].

The output is retrieved at an arbitrary point $l$.

As can be seen from Equation \eqref{eq:descreteRelativeVel} the solution for $v_\text{rel}$ is implicit. It is thus necessary to use an iterative root-finding method such as Newton-Raphson [source]

\begin{equation}\label{eq:newtonRaphson}
    f(v_\text{rel})^{i+1} = f(v_\text{rel})^i - \frac{f(v_\text{rel})^i}{f'(v_\text{rel})^i}.
\end{equation}
This process continues until 

\begin{equation} \nonumber
    f^{i+1}-f^i < \epsilon,
\end{equation}
where $\epsilon = 10^{-4}$ is an arbitrary threshold. 
\subsection{Plate}
A solved finite-difference scheme for Equation \eqref{eq:plateFDS} is presented by Equation \eqref{eq:plateImplementation} in Appendix A. A visualisation of this FDS can be found in Figure \ref{fig:plateFDS}. 

We chose the number of points to be $N_x = 20$ and $N_y = 10$ as this was found to be a great speed/quality tradeoff. 
\subsection{Connections}

\subsection{Tuning}
See \cite{Bilbao2009:NumericalSoundSynthesis} section 7.6.

\section{Instruments}\label{sec:instruments}

In this section, several configurations of PMs and connections that are inspired by real-life instruments will be shown.

In the implementation, the string-elements are subdivided into three types: bowed, plucked and sympathetic strings. These will all be connected to one plate which simulates the body of the instrument.

The user can control the plate stiffness and increase it for more interaction between the strings.

Apart from some parameters controlled by the mouse, the instruments are fully controlled by two Sensel Morphs (further referred to as Sensels).   

Videos and sound examples can be found through the following link:...

\subsection{Bowed Sitar}
The elements it consists of are: 2 bowed strings, 5 plucked strings and 13 sympathetic strings all connected to one plate.

\subsection{Hurdy Gurdy}
The hurdy gurdy from the \textbf{1500s} [source] consisting of bowed and sympathetic/plucked strings. The bowing happens through a 'resined' wheel attached to a crank that bowes these strings as the crank is turned. The pitch of about half of the bowed strings can be changed using buttons that press on the strings at different positions.

Our implementation consists of 5 bowed strings subdivided into 2 drone strings tuned to A2, E3 and 3 melody strings A3, E4 and A4 and 13 sympathetic strings all connected to one plate. 

The Sensel is vertically subdivided into 5 rows that control 5 separate 'wheels' for each individual bowed string. The bowing velocity depends on the pressure by which the Sensel is pressed. The pitch (in the model this is the wave-speed $\gamma$) of the melody strings are changed by a midi controller.

\subsection{Dulcimer}
The dulcimer is an instrument that can be seen as an 'open piano' where the musician has the hammers in their hand like drumsticks. Just like the piano, the strings are grouped in pairs or triples %triplets?
that are played simultaneously. 

In our implementation, we have 40 hammered strings (which are plucked strings with a longer excitation length)


User-controlled variables:
\begin{itemize}
    \item Bowing position
    \item Bow force
    \item Bow velocity
    \item Connection points
    \item Finger position (pitch)
\end{itemize}

The vertical velocity of the finger is linked to the bow velocity with a maximum of $V_\text{B} = 0.2$ m/s and the finger force is linked to the excitation function with a maximum of $100$ m/s$^2$.

\subsection{Sensel Morph}
Something about the sensel morph
\subsubsection{Mapping strategies}
Something about the different prototype mappings, and the "final" mapping 

\section{Discussion}\label{sec:discussion}


\section{Conclusion and Future Work}\label{sec:conclusion}

\begin{acknowledgments}
We would like to thank...
\end{acknowledgments} 

%%%%%%%%%%%%%%%%%%%%%%%%%%%%%%%%%%%%%%%%%%%%%%%%%%%%%%%%%%%%%%%%%%%%%%%%%%%%%
%bibliography here
\bibliography{smc2019bib}

\section{Appendix A}
\subsection{Finite-Difference Scheme String}
Solving Equation \eqref{eq:stiffStringFDS} for $u_l^{n+1}$ we obtain
\begin{equation}
    \begin{aligned}\label{eq:stringImplementation}
        (1 &+ \sigma_0k)u_l^{n+1} = 2u_l^n - (1 - \sigma_0k - 2\psi) u_l^{n-1} \\
        & +(\lambda^2 + \psi)(u_{l+1}^n - 2u_l^n + u_{l-1}^n)\\
        &- \mu^2(u_{l+2}^n - 4u_{l+1}^n + 6u_l^n - 4u_{l-1}^n + u_{l-2}^n) \\
        &+ \psi(u_{l+1}^{n-1} + u_{l-1}^{n-1})\\
        &-k^2\delta_{l_e}F_\text{B}\phi(v_\text{rel}),
    \end{aligned}
\end{equation}
where
\begin{equation}\nonumber
\lambda = \frac{\gamma k}{h}, \;\; \mu =  \frac{\kappa k}{h^2}, \;\; \psi = \frac{2\sigma_1k}{h^2} \;\; \text{and} \;\; \delta_{l_\text{e}} = \delta(x-x_{l_\text{e}}). 
\end{equation}
We can iteratively solve $v_\text{rel}$ using \eqref{eq:newtonRaphson}:
\begin{equation}
    \begin{aligned}
    f(v_\text{rel}) = &F_\text{B} \sqrt{2a} v_\text{rel}e^{-a v_\text{rel}^2 + 1/2} \\
    &+ \frac{2 v_\text{rel}}{k} + 2 \sigma_0 v_\text{rel} + b,
    \end{aligned}
\end{equation}
and the derivative with respect to $v_\text{rel}$ is
\begin{equation}
    \begin{aligned}
    f'(v_\text{rel}) = &F_\text{B} \sqrt{2a}(1 - 2a v_\text{rel}^2) e^{-a v_\text{rel}^2 + 1/2} \\
    &+ \frac{2}{k} + 2 \sigma_0,
    \end{aligned}
\end{equation}
where $a$ is a free variable (in our implementation set to $100$) and
\begin{equation}
    \begin{aligned}
    b =&\frac{2}{k}v_\text{B} - \frac{2}{k^2}(u_{x_\text{B}}^n - u_{x_\text{B}}^{n-1}) - \frac{\gamma^2}{h^2} (u_{x_\text{B}+1}^n - 2 u_{x_\text{B}}^n + u_{x_\text{B} - 1}^n)\\
    &+ \frac{k^2}{h^2} (u_{x_\text{B}+2}^n - 4 u_{x_\text{B}+1}^n + 6 u_{x_\text{B}}^n - 4 u_{x_\text{B}-1}^n + u_{x_\text{B}-2}^n)\\
    &+ 2 \sigma_0 v_\text{B} - \frac{2 \sigma_1}{kh^2}((u_{x_\text{B}+1}^n - 2 u_{x_\text{B}}^n + u_{x_\text{B}-1}^n) \\
    &- (u_{x_\text{B}+1}^{n-1} - 2 u_{x_\text{B}}^{n-1} + u_{x_\text{B}-1}^{n-1})).
    \end{aligned}
\end{equation}
\subsection{Finite-Difference Scheme Plate}
Solving Equation \eqref{eq:plateFDS} for $u_l^{n+1}$ we obtain
\begin{equation}
    \begin{aligned}\label{eq:plateImplementation}
        (1& + \sigma_0k)u_{l,m}^{n+1} = (2 - 4\psi + 20\mu^2)u_{l,m}^n\\
    &-\mu^2 (u_{l,m + 2}^n + u_{l,m - 2}^n + u_{l + 2,m}^n + u_{l - 2,m}^n)\\
    &- 2\mu^2(u_{l + 1,m + 1}^n + u_{l + 1,m - 1}^n + u_{l - 1,m + 1}^n + u_{l - 1,m - 1}^n)\\
    &+ (8\mu^2 + \psi)(u_{l,m + 1}^n + u_{l,m - 1}^n + u_{l + 1,m}^n + u_{l - 1,m}^n)\\
    &+ (\sigma_0 k - 1 + 4\psi) u_{l,m}^{n-1}\\
    &- \psi(u_{l,m + 1}^{n-1} + u_{l,m - 1}^{n-1} + u_{l + 1,m}^{n-1} + u_{l - 1,m}^{n-1})\\
    &+ k^2 \delta_{l_\text{e}, m_\text{e}}F_\text{e},
\end{aligned}
\end{equation}
where
\begin{equation}\nonumber
    \mu = \frac{\kappa k}{h^2} \text{,} \quad \psi = \frac{2\sigma_1k}{h^2} \quad \text{and} \quad \delta_{l_\text{e}, m_\text{e}} = \delta(x - x_{l_\text{e}}, y - y_{m_\text{e}}).
\end{equation}

\end{document}
